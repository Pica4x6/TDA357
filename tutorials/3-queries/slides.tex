\documentclass{beamer}
\usepackage[latin1]{inputenc}
\usepackage{amsmath}
\usepackage{minted}
\usepackage{breqn}
\usetheme{Warsaw}
\title[Tutorial 3]{Relational algebra and queries}
\author{Gr\'egoire D\'etrez}
\date{Jule 13, 2007}
\begin{document}

\begin{frame}
  \titlepage
\end{frame}

\section{Exercise 1: Air trafic}

\begin{frame}[fragile]{Domain Description}
\begin{verbatim}
  Airports(code,city)
  FlightCodes(code, airlineName)
  Flights(origin, destination, departure, arrival, code)
    origin -> Airports.code
    destination -> Airports.code
    code -> FlightCodes.code
\end{verbatim}

The listed flight code is the prime flight (i.e. the one used by the operating
airline). For simplicity, we assume that $departure$ and $arrival$
are integers denoting full hours, all in the same time zone, and that
$0 \le departure < arrival < 24$.
\end{frame}

\begin{frame}{Question 1}
Using this schema, write an SQL query that finds all airports that have
departures or arrivals (or both) of flights operated by Lufthansa or SAS (or
both).
\end{frame}

\begin{frame}{Question 2}
Using the schema, write an SQL query that shows the names of all cities
together with the number of flights that depart from them, and sorts them by
the number of flights in descending order (i.e. the city with the largest
number of departures first).
\end{frame}

\begin{frame}{Question 3}
Using the above schema, write a view that lists all connections from any city X
to any other city Y involving 1 or 2 legs (i.e. separate flights between two
cities: if you fly from Gothenburg to Paris with a change in Frankfurt, the
connection has two legs).

The query must return a table with the following information (and nothing
else):
\begin{itemize}
  \item the departure city X and destination city Y
  \item the departure time from X and the arrival time in Y
  \item the number of legs
  \item the total time from departure in X to arrival in Y
  \item the total time spent in the air
  \end{itemize}

A change is possible if and only if

\begin{enumerate}
  \item it happens at the same airport,
  \item the changing time is at least 1 hour, and
  \item the connecting flight is on the same day.
\end{enumerate}
\end{frame}

\begin{frame}{Question 4}
Express the query of question 1 by a relational algebra expression.
\end{frame}

\begin{frame}{Question 5}
Translate the following relational algebra expression to an SQL query:

\begin{multline}
  \pi_{First.departure,Second.arrival}( \\
    \rho_{First}(Flights) \\
    \bowtie_{First.destination = Second.origin} \\
    \rho_{Second}(Flights) \\
)
\end{multline}
\end{frame}

\section{Exercise 2: Music website}

\begin{frame}[fragile]{Domain Description}
  \footnotesize
\begin{columns}
\begin{column}{0.5\textwidth}
\begin{verbatim}
Tracks(trackId,title, length)
    length > 0
Artists(artistId, name)
Albums(albumId,title, yearReleased)
TracksOnAlbum(album,trackNr,track)
    album ->  Albums.albumId
    track ->  Tracks.trackId
    (album,track) unique
    trackNr > 0
Participates(track, artist)
    track ->  Tracks.trackId
    artist ->  Artists.artistId
\end{verbatim}
  \end{column}
  \begin{column}{0.5\textwidth}  %%<--- here
\begin{verbatim}
Users(username, email, name)
    email unique
Playlists(user, playlistName)
    user ->  Users.username
InList(user, playlist, number,track)
    (user, playlist) ->  Playlists.(user, playlistName)
    track ->  Tracks.trackId
PlayLog(user,time,track)
    user ->  Users.username
    track ->  Tracks.trackId
    (user,time) unique
\end{verbatim}
  \end{column}
\end{columns}
\end{frame}

\begin{frame}{Domain Description (cont.)}

An artist can be either a solo artist or a group, the design makes no
difference between the two kinds. Tracks are recorded by one or more artists,
and each track can appear on one or more albums (but no more than once on each
album) to account for e.g. ``Greatest hits'' or collection albums.

Users of the site can register, in order to create playlists, which are simply
ordered collections of tracks.

Finally, the system stores a log over all songs played by registered users, to
calculate statistics and to give suggestions and feedback.  (Note: The actual
music files to be streamed is considered to be stored separately, outside the
scope of this schema.)
\end{frame}

\begin{frame}
  \frametitle{Question 1}
  Write an SQL query that lists all artists appearing on any album released this
  year (2016).
\end{frame}

\begin{frame}
  \frametitle{Question 2}

 % 
 % **Question 2**
 % Write an SQL query that lists, for each user, how many playlists that user has.
\end{frame}

\begin{frame}
  \frametitle{Question 3}
 % 
 % **Question 3**
 % Write an SQL query that lists, for each track, its ``trackId`` and title,
 % together with the number of times that track has been played, and the number of
 % distinct users that have played it.
\end{frame}

\begin{frame}
  \frametitle{Question 4}
  Write an SQL query that finds the title, length and album title of the longest
  track in the database. If the track appears on more than one album, list the
  album where it appeared first. If more than one track of the same length
  qualifies, list the one that was released first, as given by the album it
  appears on. If there is still a tie, list all such tracks.
\end{frame}

\begin{frame}
  \frametitle{Question 5}
  What does the following relational algebra expression compute (answer in
  plain text):
  
  $$
    \tau_x(
      \gamma_{playlistName,COUNT(*)\rightarrow x}(
        \sigma_{playlistName=playlist}(Playlists \times InList)
      )
    )
  $$
\end{frame}

\begin{frame}
  \frametitle{Question 6}
  Translate the following relational algebra expression(s) to corresponding SQL:
  $$ let\ R1 = \gamma_{user,track,COUNT(*)\rightarrow nrTimes}(PlayLog) $$
  $$ \sigma_{avgNrTimes>=10}(\gamma_{track,AVG(nrTimes)\rightarrow avgNrTimes}(R1)) $$
\end{frame}

\begin{frame}[fragile]
  \frametitle{Question 7}
  Translate the following SQL query to relational algebra:
\footnotesize
\begin{minted}{sql}
SELECT album, MAX(trackNr) AS nrOfTracks, SUM(length) AS totalLength
FROM Albums, TracksOnAlbum, Tracks
WHERE albumId = album
AND trackId = track
GROUP BY albumId
ORDER BY totalLength DESC;
\end{minted}
\end{frame}

\begin{frame}
  \frametitle{Question 8}
  Write a relational algebra expression that lists the artist(s) appearing in the
  highest number of distinct playlists. In case of a tie for highest number of
  different playlists, list all such artists.
\end{frame}

\end{document}
