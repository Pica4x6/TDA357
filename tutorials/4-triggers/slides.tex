\documentclass{beamer}
\usepackage[latin1]{inputenc}
\usepackage{amsmath}
\usetheme{Warsaw}
\title[Tutorial 4]{Triggers}
\author{Gr\'egoire D\'etrez}
\date{February 24, 2016}

\renewcommand{\sectionname}{Exercice}

\begin{document}
\frame{\titlepage}


% ~~~ Exercice 1 ~~~~~~~~~~~~~~~~~~~~~~~~~~~~~~~~~~~~~~~~~~~~~~~~~~~~~~~~~~~~~
\section{Booking flights}
\frame{\sectionpage}

\begin{frame}[fragile]
  \frametitle{Domain Description}
  We extend the shema from last week with the following relations to handle
  bookings:
\footnotesize
\begin{verbatim}
AvailableFlights(_flight_, _date_, numberOfFreeSeats, price)
  flight -> Flights.code
Bookings(_reference_, flight, date, passenger, price)
  (flight, date) -> AvailableFlights.(code, date)
\end{verbatim}
\normalsize
Where \verb|Bookings.price| is the price that was paid for a particular ticket
and \verb|AvailableFlights.price| is the current price of the flight, which is
different.
\end{frame}

\begin{frame}{Question 1}
  Create a view that lists booking references, passengers, flight codes, dates,
  and departure and destination cities.
\end{frame}

\begin{frame}{Question 2}
  Create a trigger on the view defined in question 1. This trigger takes care
  of booking a new passenger to a flight. It is fired by an insertion of a
  passenger and a flight code to the view, for instance, ``book Annie Adams for
  AF666''. Its effect should be the following:
  \begin{enumerate}
    \item if the number of free seats on AF666 is positive, decrement it by
      one; the booking is successful
    \item if there are no free seats, the booking fails
    \item if the booking is successful, add Annie Adams and AF666 to Bookings,
      with the price given in AvailableFlights when booking her; also add a
      booking reference which is the maximum of the previous references (for
      all flights) plus one
    \item if the booking is successful, increment the price by 50 SEK for the
      next passenger (thus the fuller the flight, the more you pay)
  \end{enumerate}
\end{frame}

\begin{frame}[fragile]
  \frametitle{Refactoring}
  The airline decides to upgrade its database to keep track of its fleet. This
  means adding a table to list the available planes and updating the
  \texttt{AvailableFlights} in the following way:
\footnotesize
\begin{verbatim}
Planes(_regnr_, capacity)
AvailableFlights(_flight_, _date_, numberOfFreeSeats, price, plane)
  (flight, date) -> Flights.(code, date)
    plane -> Planes.regnr
\end{verbatim}
\normalsize
\end{frame}

\begin{frame}[fragile]
  \frametitle{Question 3}
  Your job is to create a trigger that automatically update
  \verb|numberOfSeats| when the company changes plane for a flight.
\end{frame}


% ~~~ Exercice 2 ~~~~~~~~~~~~~~~~~~~~~~~~~~~~~~~~~~~~~~~~~~~~~~~~~~~~~~~~~~~~~
\section{At the restaurant}
\frame{\sectionpage}

\begin{frame}[fragile]
  \frametitle{Domain Description}
In this exercise, we are creating the following database for a
restaurant:

\begin{verbatim}
Tables(_number_, seats)
Bookings(_name_, _time_, nbpeople, table)
  table -> Tables.number
\end{verbatim}

For the sake of simplicity, we assume that the database only needs to
hold bookings for the current day and that bookings are always done on
the hour (i.e. \texttt{time} is an integer between 0 and 23).

Finally, tables are always booked for two hours. This means that if
table 1 is booked at 19.00, it can't be booked at 20.00 but it can be
booked again at 21.00.
\end{frame}

\begin{frame}
  \frametitle{Question 1}
  Write a view that lists the times at which tables are blocked by a booking.
  In the example above, where table 1 is booked at 19.00, the view should
  contain the following rows:
  \begin{center}
    \begin{tabular}{l|l}
      table & time\\
      \hline
      1 & 18\\
      1 & 19\\
      1 & 20\\
    \end{tabular}
  \end{center}
\end{frame}

\begin{frame}
  \frametitle{Question 2}
  Write a trigger on the table \texttt{Bookings} that automatically assign a
  table to new rows if none is specified. The assigned table should respect the
  following rules:
  \begin{itemize}
    \item it should be free for the duration of the booking.
    \item it should be big enough for the number of people in the party
    \item it should be the smallest possible table to accommodate this number
      of people
  \end{itemize}
\end{frame}


% ~~~ Exercice 3 ~~~~~~~~~~~~~~~~~~~~~~~~~~~~~~~~~~~~~~~~~~~~~~~~~~~~~~~~~~~~~
\section{Wiki}\label{exercise-3-wiki}
\frame{\sectionpage}

\begin{frame}[fragile]
  \frametitle{Domain Description}
  In this exercise, we are creating a database for a wiki. A wiki is a website
  that \emph{allows collaborative modification of its content and structure
  directly from the web browser} (Wikipedia).

  To make collaborative edition easier, we needs to keep an history of the
  modifications of each page. To achieve this, we will use a simple model
  that simply keeps each version of each page as a separate row:

\begin{verbatim}
PageRevision(_name_, _date_, author, text)
\end{verbatim}
\end{frame}

\begin{frame}
  \frametitle{Question 1}
  To make it easier to access the wiki, creat a view \texttt{Page(name,\
  last\_author,\ text)} that shows only the latest version of each page.
\end{frame}

\begin{frame}
  \frametitle{Question 2}
Create a trigger on your newly created view so that
when a user tries to update a given page, a new revision is created
instead.
\end{frame}

\begin{frame}[fragile]
Sometimes, pages on the wiki needs to be completely deleted (for
instance, if a page contains sensitive information or copyrighted
content). In that case, we want to remove all revisions of the page from
the database but we still want to remember that the page has existed but
has been deleted. To this end, we add the following table to our
database:

\begin{verbatim}
DeleteLog(_pagename_, _date_)
\end{verbatim}
\end{frame}

\begin{frame}
  \frametitle{Question 3}
  Write a trigger on the \texttt{Page} view such that when a page is deleted:
  \begin{itemize}
    \item all its revisions are removed from the database
    \item the deletion is recorded in the \texttt{DeleteLog}.
  \end{itemize}
\end{frame}


\end{document}
